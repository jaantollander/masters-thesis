\degreeprogram{Computer, Communication and Information Sciences}
\major{Computer Science}
\code{SCI3042}
\univdegree{MSc}
\thesisauthor{Jaan Tollander de Balsch}
\thesistitle{Monitoring parallel file system usage in a high-performance computer cluster}
\place{Espoo}
%\date{dd.mm.yyyy}
\supervisor{Prof. Petteri Kaski}
\advisor{Dr. Sami Ilvonen}
\uselogo{aaltoRed}{''}
\keywords{monitoring computer systems\spc observability\spc computer cluster\spc high-performance computing\spc parallel file system\spc Lustre\spc I/O behavior\spc time series analysis\spc exploratory data analysis}
\thesisabstract{
Many high-performance computer clusters, rely on a system-wide, shared, parallel file system for large storage capacity and bandwidth.
A shared file system is available across the entire system, making it user-friendly but prone to problems from heavy use.
Such use can cause congestion and slow down or even halt the whole system, harming all users who use the parallel file system.
In this thesis, we investigate whether monitoring file system usage in a production system at CSC can help identify the causes of slowdowns, such as specific users or jobs.
The long-goal at CSC is to build an automatic, real-time monitoring and warning system that system administrators can use to make decisions on alleviating the slowdowns.
Specifically, we monitor the usage of the Lustre parallel file system with Lustre Jobstats feature in the Puhti cluster, which is a petascale cluster with a diverse user base.
We explain the necessary details of the Puhti cluster and our monitoring system to understand the Lustre file system usage data.
During the thesis, we discovered issues in the data quality from Lustre Jobstats.
The issues affected identifiers in the data, making some data unreliable and limiting our ability to build an automatic, real-time analysis.
Nevertheless, we obtained a feasible data set for explorative batch analysis.
We demonstrate 24 hours of monitoring data by visually demonstrating file system usage patterns at low and high-level.
Furthermore, we show that we can use file system usage data to identify causes of relative changes in I/O trends, particularly large relative increases.
Finally, we explore ideas for future work on monitoring file system usage with reliable data from longer periods.
}
\copyrighttext{This work is licensed under the Creative Commons Attribution 4.0 International (CC BY 4.0) license.}
{This work is licensed under the Creative Commons Attribution 4.0 International (CC BY 4.0) license.}

% customization
\usepackage{subcaption}
\DeclareCaptionFormat{custom}
{%
    \textbf{#1#2} \small{#3}
}
\captionsetup{format=custom}
\captionsetup{width=0.9\textwidth,format=custom}
