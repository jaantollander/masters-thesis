\makecoverpage
\makecopyrightpage

% English abstract
\begin{abstractpage}[english]
\abstracttext
\end{abstractpage}


% Finnish abstract
\degreeprogram{Tieto-, tietoliikenne- ja informaatiotekniikka}
\major{Tietotekniikka}
\thesistitle{Rinnakkaistiedostojärjestelmän käytön valvonta suurteholaskenta tietokoneklusterissa}
\advisor{TkT. Sami Ilvonen}
\keywords{tietokonejärjestelmien valvonta\spc havaittavuus\spc tietokoneklusteri\spc suurteholaskenta\spc rinnakkaistiedostojärjestelmä\spc Lustre\spc I/O käyttäytyminen\spc aikasarjaanalyysi\spc tutkiva data-analyysi}

\begin{abstractpage}[finnish]
Monet tehokkaat tietokoneklusterit luottavat järjestelmän laajuiseen, jaettuun rinnakkaistiedostojärjestelmään suuren tallennuskapasiteetin ja kaistanleveyden saavuttamiseksi.
Jaettu tiedostojärjestelmä on käytettävissä koko järjestelmässä, mikä tekee siitä käyttäjäystävällisen, mutta altis raskaan käytön aiheuttamille ongelmille.
Tällainen käyttö voi aiheuttaa ruuhkautumista ja hidastaa tai jopa pysäyttää koko järjestelmän, mikä vahingoittaa kaikkia rinnakkaistiedostojärjestelmää käyttäviä käyttäjiä.
Tässä opinnäytetyössä tutkimme voiko CSC:n tuotantojärjestelmän tiedostojärjestelmän käytön seuranta auttaa tunnistamaan hidastumisen syitä, kuten tiettyjä käyttäjiä tai töitä.
CSC:n pitkäaikainen tavoite on rakentaa automaattinen, reaaliaikainen valvonta- ja varoitusjärjestelmä, jonka avulla järjestelmänvalvojat voivat tehdä päätöksiä hidastumisen lievittämiseksi.
Tarkemmin sanottuna seuraamme Lustre rinnakkaistiedostojärjestelmän käyttöä Lustre Jobstatsilla Puhti-klusterissa, joka on monipuolisen käyttäjäkunnan omaava petascale-klusteri.
Selitämme tarvittavat yksityiskohdat Puhti-klusterista ja valvontajärjestelmästämme Lustre-tiedostojärjestelmän käyttötietojen ymmärtämiseksi.
Opinnäytetyön aikana havaitsimme ongelmia Lustre Jobstats:in tietojen laadussa.
Ongelmat vaikuttivat tiedoissa oleviin tunnisteisiin, mikä teki joistakin tiedoista epäluotettavia ja rajoitti kykyämme luoda automaattinen, reaaliaikainen analyysi.
Siitä huolimatta saimme käyttökelpoisen tietojoukon tutkivaa eräanalyysiä varten.
Esittelemme 24 tunnin seurantatietoja näyttämällä visuaalisesti tiedostojärjestelmän käyttötapoja matalalla ja korkealla tasolla.
Lisäksi osoitamme, että voimme käyttää tiedostojärjestelmän käyttötietoja tunnistamaan syitä suhteellisiin muutoksiin I/O-trendeissä, erityisesti suurissa suhteellisissa nousuissa.
Lopuksi tutkimme ideoita tulevaa työtä varten tiedostojärjestelmän käytön seuraamiseksi luotettavalla tiedolla pidemmältä ajalta.
\end{abstractpage}


% Acknowledgments
\addcontentsline{toc}{section}{Acknowledgments}
\section*{Acknowledgments}
I offer my deepest gratitude to my thesis advisor \emph{Sami Ilvonen}, whose work with the monitoring system and writing advice was instrumental in completing this thesis.
Furthermore, I offer my sincerest gratitude to my supervising professor \emph{Petteri Kaski}, whose guidance with writing improved the thesis significantly.

I thank \emph{Sebastian Von Alfthan} for providing me the chance to work as part of an extraordinary team on a fascinating problem; \emph{Simon Westersund}, who set up the monitoring system; \emph{Ulf Tigerstedt}, and who helped to manage the database;
\emph{CSC - The IT Center for Science} as a place to work with amazing and talented people; \emph{Aalto University} for the many years of education; and the \emph{Aalto Scientific Computing} admins, \emph{Simo Tuomisto} and \emph{Mikko Hakala}, for explaining their monitoring workflow.

\vspace{5cm}
Espoo \today

\vspace{5mm}
{\hfill Jaan Tollander de Balsch \hspace{1cm}}

\newpage

\setcounter{tocdepth}{2}
\addcontentsline{toc}{section}{Contents}
\tableofcontents
