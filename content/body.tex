\makecoverpage
\makecopyrightpage

% English abstract
\begin{abstractpage}[english]
\abstracttext
\end{abstractpage}


% Finnish abstract
%\degreeprogram{Tieto-, tietoliikenne- ja informaatiotekniikka}
%\major{Tietotekniikka}
\thesistitle{Rinnakkaistiedostojärjestelmän käytön valvonta suurteholaskenta tietokoneklusterissa}
\advisor{TkT. Sami Ilvonen}
\keywords{tietokonejärjestelmien valvonta\spc havaittavuus\spc tietokoneklusteri\spc suurteholaskenta\spc rinnakkaistiedostojärjestelmä\spc Lustre\spc I/O käyttäytyminen\spc aikasarjaanalyysi\spc tutkiva data-analyysi}

\begin{abstractpage}[finnish]
Monet tehokkaat tietokoneklusterit luottavat järjestelmän laajuiseen, jaettuun rinnakkaistiedostojärjestelmään suuren tallennuskapasiteetin ja kaistanleveyden saavuttamiseksi.
Jaettu tiedostojärjestelmä on käytettävissä koko järjestelmässä, mikä tekee siitä käyttäjäystävällisen, mutta altis raskaan käytön aiheuttamille ongelmille.
Tällainen käyttö voi aiheuttaa ruuhkautumista ja hidastaa tai jopa pysäyttää koko järjestelmän, mikä vahingoittaa kaikkia rinnakkaistiedostojärjestelmää käyttäviä käyttäjiä.
Tässä opinnäytetyössä tutkimme voiko CSC:n tuotantojärjestelmän tiedostojärjestelmän käytön seuranta auttaa tunnistamaan hidastumisen syitä, kuten tiettyjä käyttäjiä tai töitä.
CSC:n pitkäaikainen tavoite on rakentaa automaattinen, reaaliaikainen valvonta- ja varoitusjärjestelmä, jonka avulla järjestelmänvalvojat voivat tehdä päätöksiä hidastumisen lievittämiseksi.
Tarkemmin sanottuna seuraamme Lustre rinnakkaistiedostojärjestelmän käyttöä Lustre Jobstats ominaisuudella Puhti-klusterissa, joka on monipuolisen käyttäjäkunnan omaava petascale-klusteri.
Selitämme tarvittavat yksityiskohdat Puhti-klusterista ja valvontajärjestelmästämme Lustre-tiedostojärjestelmän käyttötietojen ymmärtämiseksi.
Opinnäytetyön aikana havaitsimme ongelmia Lustre Jobstats:in tietojen laadussa.
Ongelmat vaikuttivat tiedoissa oleviin tunnisteisiin, mikä teki joistakin tiedoista epäluotettavia ja rajoitti kykyämme luoda automaattinen, reaaliaikainen analyysi.
Siitä huolimatta saimme käyttökelpoisen tietojoukon tutkivaa data-analyysiä varten.
Esittelemme 24 tunnin seurantatietoja näyttämällä visuaalisesti tiedostojärjestelmän käyttötapoja matalalla ja korkealla tasolla.
Lisäksi osoitamme, että voimme käyttää tiedostojärjestelmän käyttötietoja tunnistamaan syitä suhteellisiin muutoksiin I/O-trendeissä, erityisesti suurissa suhteellisissa nousuissa.
Lopuksi tutkimme ideoita tulevaa työtä varten tiedostojärjestelmän käytön seuraamiseksi luotettavalla tiedolla pidemmältä ajalta.
\end{abstractpage}


% Acknowledgments
\addcontentsline{toc}{section}{Acknowledgments}
\section*{Acknowledgments}
I sincerely thank my thesis advisor, doctor Sami Ilvonen and supervisor, professor Petteri Kaski.
Sami's work with the monitoring system and writing advice were instrumental in completing this thesis.
Petteri's guidance with writing and thoughtful comments improved the thesis significantly.

Furthermore, I thank Sebastian Von Alfthan for providing me the chance to work as part of a great team on a fascinating problem; 
Simon Westersund, who set up the monitoring system and proof-reading the thesis; 
Ulf Tigerstedt, who helped to manage the database; 
CSC - The IT Center for Science for a place to work with amazing and talented people; 
Aalto University for the many years of education; 
and the Aalto Scientific Computing system administrators, Simo Tuomisto and Mikko Hakala, for explaining their monitoring workflow.

\vspace{5cm}
Espoo \today

\vspace{5mm}
{\hfill Jaan Tollander de Balsch \hspace{1cm}}

\newpage

\setcounter{tocdepth}{2}
\addcontentsline{toc}{section}{Contents}
\tableofcontents
